\documentclass{report}

\usepackage[T2A]{fontenc}
\usepackage[utf8]{luainputenc}
\usepackage[english, russian]{babel}
\usepackage[pdftex]{hyperref}
\usepackage[14pt]{extsizes}
\usepackage{listings}
\usepackage{color}
\usepackage{geometry}
\usepackage{enumitem}
\usepackage{multirow}
\usepackage{graphicx}
\usepackage{indentfirst}

\geometry{a4paper,top=2cm,bottom=3cm,left=2cm,right=1.5cm}
\setlength{\parskip}{0.5cm}
\setlist{nolistsep, itemsep=0.3cm,parsep=0pt}

\lstset{language=C++,
basicstyle=\footnotesize,
keywordstyle=\color{blue}\ttfamily,
stringstyle=\color{red}\ttfamily,
commentstyle=\color{green}\ttfamily,
morecomment=[l][\color{magenta}]{\#},
tabsize=4,
breaklines=true,
breakatwhitespace=true,
title=\lstname,
}

\makeatletter
\renewcommand\@biblabel[1]{#1.\hfil}
\makeatother

\begin{document}

\begin{titlepage}

\begin{center}
Министерство науки и высшего образования Российской Федерации
\end{center}

\begin{center}
Федеральное государственное автономное образовательное учреждение высшего образования \\
Национальный исследовательский Нижегородский государственный университет им. Н.И. Лобачевского
\end{center}

\begin{center}
Институт информационных технологий, математики и механики
\end{center}

\vspace{4em}

\begin{center}
\textbf{\LargeОтчет по лабораторной работе} \\
\end{center}
\begin{center}
\textbf{\Large«Построение выпуклой оболочки для компонент бинарного изображения»} \\
\end{center}

\vspace{4em}

\newbox{\lbox}
\savebox{\lbox}{\hbox{text}}
\newlength{\maxl}
\setlength{\maxl}{\wd\lbox}
\hfill\parbox{7cm}{
\hspace*{5cm}\hspace*{-5cm}\textbf{Выполнил:} \\ студент группы 381808-1 \\ Кустова Анастасия\\
\\
\hspace*{5cm}\hspace*{-5cm}\textbf{Проверил:}\\ доцент кафедры МОСТ, \\ кандидат технических наук \\ Сысоев А.В.\\
}
\vspace{\fill}

\begin{center} Нижний Новгород \\ 2020 \end{center}

\end{titlepage}

\setcounter{page}{2}

\newpage

% Введение
\section*{Введение}
\addcontentsline{toc}{section}{Введение}
Задача построения выпуклых оболочек, является одной из центральных задач вычислительной геометрии. Важность этой задачи происходит не только из-за огромного количества приложений (в распознавании образов, обработке изображений, базах данных, в задаче раскроя и компоновки материалов, математической статистике), но также и из-за полезности выпуклой оболочки как инструмента решения множества задач вычислительной геометрии. Очень широко алгоритмы построения выпуклой оболочки используются в геоинформатике и геоинформационных системах. Выпуклую оболочку объекта можно строить различными аналитическими способами, которые в основном являются сложными. Ниже предлагается один из способов построения выпуклой оболочки на языке C++ с использованием технологий MPI.

\newpage

% Постановка задачи
\section*{Постановка задачи}
\addcontentsline{toc}{section}{Постановка задачи}
В данной работе необходимо реализовать построение выпуклой оболочки для компонент бинарного изображения. Данная работа должна содержать последовательную и параллельную версии. 
\par
На основе разработанной программы необходимо провести вычислительные эксперименты, сравнив время работы параллельного и последовательного алгоритмов на разных входных данных и разном количестве процессов, затем сделать вывод об эффективности параллельной версии.
\par
Также следует экспериментально проверить корректность разработанных версий алгоритма с помощью Google тестов и проиллюстрировать работу на реальных изображениях.
\newpage

% Описание алгоритма
\section*{Описание алгоритма}
\addcontentsline{toc}{section}{Описание алгоритма}
1) Для начала необходимо выделить компоненты бинарного изображения.
\par
Обход по изображению выполняется из верхнего левого угла. На внешнем цикле – перебор строк, на внутреннем – перебор столбцов строки. Точки с цветом фона опускаем. Нетрудно заметить, что при таком подходе всегда уже обработаны соседи сверху и слева.
\begin{lstlisting}
if A = O
	do nothing
else if (not B labeled) and (not C labeled)
	increment label numbering and label A
else if B xor C labeled
	copy label to A
else if B and C labeled
	if B label = C label
		copy label to A
	else
		copy either B label or C label to A
record equivalence of labels
\end{lstlisting}
\par 
После обхода всего изображения, следует его повторить и осуществить повторное выделение областей, учитывающее эквивалентности областей.
\par 
2) На втором шаге необходимо для каждой компоненты построить выпуклую оболочку.
\par 
Построение этой оболочки осуществляется с помощью алгоритма Джарвиса, на вход которого подаётся вектор координат ненулевых (чёрных) пикселей. Описание алгоритма приведено ниже.
\begin{center}
Алгоритм Джарвиса
\end{center}
\par 
Пусть на плоскости задано конечное множество точек A. Оболочкой этого множества называется любая замкнутая линия H без самопересечений такая, что все точки из A лежат внутри этой кривой. Если кривая H является выпуклой (например, любая касательная к этой кривой не пересекает ее больше ни в одной точке), то соответствующая оболочка также называется выпуклой. Минимальной выпуклой оболочкой называется выпуклая оболочка минимальной длины (минимального периметра).
\par 
Возьмем точку p0 нашего множества с самой маленькой у-координатой (если таких несколько, берем самую правую из них). Добавляем ее в ответ. На каждом следующем шаге для последнего добавленного pi ищем pi+1 среди всех недобавленных точек и p0 с максимальным полярным углом относительно pi (Если углы равны, надо сравнивать по расстоянию). Добавляем pi+1 в ответ. Если pi+1==p0 , заканчиваем алгоритм.
\par
3) На третьем шаге в случае работы с реальным изображением необходимо соединить линиями точки, образующие минимальные выпуклые оболочки.
\par
Из результата работы предыдущего алгоритма получаем вектор координат точек, образующих выпуклую оболочку. В этом векторе точки уже находятся в том порядке, в котором их нужно соединить.
\newpage

% Схема распараллеливания
\section*{Схема распараллеливания}
\addcontentsline{toc}{section}{Схема распараллеливания}
На первом шаге – выделение компонент бинарного изображения, каждому процессу достаётся несколько строк матрицы изображения, с которыми он будет работать. Далее каждый процесс отмечает на этих строках связанные компоненты и затем это всё собирается с помощью Allgather в новое изображение. Затем строится граф смежности, определяющий, какие участки изображения относятся к одной компоненте. Далее осуществляется повторное выделение областей, учитывающее эквивалентности областей – изображение разбивается на отдельные участки и каждому процессу достаётся несколько строк матрицы, которые затем собираются в результирующее изображение с помощью Allgather.
\par
На втором шаге – алгоритм Джарвиса, вектор, содержащий координаты точек, распределяется с помощью Scatter между всеми процессами. Затем, когда ищутся самые левые и самые нижние точки, то сначала ищутся минимумы на участке у каждого процесса, а затем, эти результаты собираются с помощью Gather на нулевом процессе и там снова ищется минимум. И с помощью Broadcast о результате операции становится известно всем остальным процессам.
\newpage

% Описание программной реализации
\section*{Описание программной реализации}
\addcontentsline{toc}{section}{Описание программной реализации}
В программе имеются следующие функции и структуры:
\begin{itemize}
\item Структура, описывающая координаты пикселя: 
\end{itemize}
\begin{lstlisting}
	struct Point;
\end{lstlisting}
\begin{itemize}
\item Функция, выводящая пиксели изображения через символы на экран (не предназначен для работы с реальным изображением): 
\end{itemize}
\begin{lstlisting}
	void printImage(int ** image, int height, int width);
\end{lstlisting}
\begin{itemize}
\item Функция, которая отмечает на изображении связанные компоненты и выводит количество компонент:
\end{itemize}
\begin{lstlisting}
	std::vector<int> searchConnectedComponents(int ** image, int height, int width);
\end{lstlisting}
\begin{itemize}
\item Функция, которая для каждой компоненты записывает в вектор координаты не нулевых (чёрных пикселей), далее этот список передаёт в функцию buildHull, которая строит выпуклую оболочку: 
\end{itemize}
\begin{lstlisting}
	int ** buildComponentConvexHull(int ** image, int height, int width);
\end{lstlisting}
\begin{itemize}
\item Параллельная реализация алгоритма Джарвиса по построению выпуклой оболочки: 
\end{itemize}
\begin{lstlisting}
	std::vector<Point> buildHull(std::vector<Point> m_set);
\end{lstlisting}
\begin{itemize}
\item Последовательная реализация алгоритма Джарвиса по построению выпуклой оболочки: 
\end{itemize}
\begin{lstlisting}
	std::vector<Point> buildHull_sequential(std::vector<Point> m_set);
\end{lstlisting}
\begin{itemize}
\item Поиск координат самой левой точки:
\end{itemize}
\begin{lstlisting}
	Point find_leftmost_point(std::vector<Point> buf, int rank, int size, int countpr); 
\end{lstlisting}
\begin{itemize}
\item Функция, возвращающая отдельные компоненты изображения:
\end{itemize}
\begin{lstlisting}
	int ** getComponent(int ** image, int height, int width, int number); 
\end{lstlisting}
\begin{itemize}
\item Функция, объединяющая отдельные компоненты в одно изображение:
\end{itemize}
\begin{lstlisting}
	int ** imageOr(int ** image1, int ** image2, int height, int width);
\end{lstlisting}
\begin{itemize}
\item Функция, рисующая выпуклую оболочку:
\end{itemize}
\begin{lstlisting}
	int ** drawLines(int ** image, int height, int width, std::vector<Point> vec);
\end{lstlisting}
\newpage

% Описание экспериментов
\section*{Описание экспериментов}
\addcontentsline{toc}{section}{Описание экспериментов}
Эксперименты проводились на 4-х ядерном компьютере с операционной системой – Windows 10, который имеет 8 логических процессоров. Результаты экспериментов приведены в таблице:
\begin{table}[!h]
\caption{Результаты вычислительных экспериментов}
\centering
\begin{tabular}{lllll}
Размер & Количество процессов & Паралл. время & Послед. время & Разница\\
500*500 & 4 & 0.18 & 0.23 & 0.05\\
500*500 & 6 & 0.215 & 0.443 & 0.228\\
500*500 & 8 & 0.137 & 0.5 & 0.363\\
1000*1000 & 3 & 1.3 & 1.457 & 0.157\\
1000*1000 & 6 & 1.5 & 1.74 & 0.24\\
1000*1000 & 8 & 1.323 & 1.776 & 0.423\\
\end{tabular}
\end{table}
\par
По результатам эксперимента, можно сделать вывод о том, что параллельная версия работает быстрее последовательной. И при увеличении размеров исходной матрицы, эта разница становится более заметной.
\par
Kорректность работы программы могут результаты гугл-тестов и наглядный пример картинки.
\newpage

% Заключение
\section*{Заключение}
\addcontentsline{toc}{section}{Заключение}
В ходе выполнения работы было успешно реализовано построение выпуклой оболочки для компонент бинарного изображения в последовательном и параллельном вариантах. Работоспособность программы была проверена с помощью библиотеки для модульного тестирования и с помощью реального изображения.
\par 
Вычислительный эксперимент показал, что параллельный вариант алгоритма предпочтительнее, так как работает быстрее последовательного.
\newpage

% Список литературы
\begin{thebibliography}{1}
\addcontentsline{toc}{section}{Список литературы}
\bibitem{Gergel}
Гергель В. П., Стронгин Р. Г. Основы параллельных вычислений для многопроцессорных вычислительных систем. – 2003.
\bibitem{parallel} Parallel: Лаборатория Параллельных информационных технологий НИВЦ МГУ [Электронный ресурс] // URL: \url {https://parallel.ru/vvv/mpi.html} (дата обращения: 28.11.2020)
\bibitem{Algolist} Algolist: Алгоритмы, методы, исходники [Электронный ресурс] // URL: \url {http://algolist.manual.ru/sort/radix_sort.php} (дата обращения: 28.11.2020)
\end{thebibliography}
\newpage

% Приложение
\section*{Приложение}
\addcontentsline{toc}{section}{Приложение}
\begin{lstlisting}
						// main.cpp

// Copyright 2020 Kustova Anastasiya
#include <gtest-mpi-listener.hpp>
#include <gtest/gtest.h>
#include <math.h>
#include <opencv/cv.h>
#include <opencv/highgui.h>
#include <opencv2/core/core.hpp>
#include <opencv2/highgui/highgui.hpp>
#include <opencv2/opencv.hpp>
#include <vector>
#include <iostream>
#include "./convex_hull_for_binary.h"
using namespace cv;

int main(int argc, char** argv){
    MPI_Init(&argc, &argv);
    int size, rank;
    MPI_Comm_rank(MPI_COMM_WORLD, &rank);
    MPI_Comm_size(MPI_COMM_WORLD, &size);

    Mat image1 = imread("/home/osboxes/Desktop/Jac/Jacoby/modules/task_3/kustova_a_convex_hull_for_binary/6.jpg", IMREAD_COLOR);
    if (rank == 0) {
        namedWindow("original", CV_WINDOW_AUTOSIZE);
        imshow("original", image1);
        waitKey(0);
    }
    Mat image2 = image1;
    int height = image1.rows;
    int width = image1.cols;
    int ** image = new int*[height];
    for (int i = 0; i < height; i++) {
        image[i] = new int[width];
    }
    for (int i = 0; i < height; i++) {
        for (int j = 0; j < width; j++) {
            if (image1.at<Vec3b>(i, j)[0] >= 200) {
                image[i][j] = 0;
            } else {
                image[i][j] = 1;
            }
        }
    }
    int ** image_s = image;
    double st = MPI_Wtime();
    int ** result = buildImageConvexHull(image, height, width, 0);
    double end = MPI_Wtime();
    double st_s = MPI_Wtime();
    buildImageConvexHull(image_s, height, width, 1);
    double end_s = MPI_Wtime();
    if (rank == 0) {
        std::cout <<"Parallel time: "<<end-st<<std::endl;
        std::cout <<"Sequential time: "<<end_s-st_s<<std::endl;
    } 
    for (int i = 0; i < height; i++) {
        for (int j = 0; j < width; j++) {
            if (result[i][j] == 0) {
                image2.at<Vec3b>(i, j)[0] = 255;
                image2.at<Vec3b>(i, j)[1] = 255;
                image2.at<Vec3b>(i, j)[2] = 255;
            } else if (result[i][j] == 1){
                image2.at<Vec3b>(i, j)[0] = 0;
                image2.at<Vec3b>(i, j)[1] = 0;
                image2.at<Vec3b>(i, j)[2] = 0;
            } else {
                image2.at<Vec3b>(i, j)[0] = 0;
                image2.at<Vec3b>(i, j)[1] = 0;
                image2.at<Vec3b>(i, j)[2] = 255;
            }
        }
    }
    if (rank == 0) {
        namedWindow("original", CV_WINDOW_AUTOSIZE);
        imshow("original", image2);
        waitKey(0);
        destroyAllWindows();
    }
    MPI_Finalize();
}
\end{lstlisting}
\begin{lstlisting}
						// convex_hull_for_binary.h

// Copyright 2020 Kustova Anastasiya
#ifndef MODULES_TASK_3_KUSTOVA_A_CONVEX_HULL_FOR_BINARY_CONVEX_HULL_FOR_BINARY_H_
#define MODULES_TASK_3_KUSTOVA_A_CONVEX_HULL_FOR_BINARY_CONVEX_HULL_FOR_BINARY_H_

#include <mpi.h>
#include <vector>
struct Point {
    int x, y;
    friend Point operator-(const Point& a, const Point& b) {
        return Point{a.x - b.x, a.y - b.y};
    }
    friend bool operator!=(const Point& a, const Point& b) {
        return !(a == b);
    }
    friend bool operator==(const Point& a, const Point& b) {
        return (a.x == b.x) && (a.y == b.y);
    }
};
void printImage(int ** image, int height, int width);
std::vector<int> searchConnectedComponents(int ** image, int height, int width);
int ** buildComponentConvexHull(int ** image, int height, int width, int type);
int ** buildImageConvexHull(int ** image, int height, int width, int type);
std::vector<Point> buildHull(std::vector<Point> m_set);
std::vector<Point> buildHull_sequential(std::vector<Point> m_set);
int ** drawLines(int ** image, int height, int width, std::vector<Point> vec);
double my_abs(double x);
Point find_leftmost_point(std::vector<Point> buf, int rank, int size, int countpr);
bool is_counter_clockwise(const Point& p, const Point& i, const Point& q);
int ** getComponent(int ** image, int height, int width, int number);
int ** imageOr(int ** image1, int ** image2, int height, int width);
std::vector<int> combineComponentParts(int ** image, int height, int width, int componentsNumber);
int giveNumbersToComponents(int ** image, int height, int width);
void printImage(int ** image, int height, int width);
#endif  // MODULES_TASK_3_KUSTOVA_A_CONVEX_HULL_FOR_BINARY_CONVEX_HULL_FOR_BINARY_H_
\end{lstlisting}
\begin{lstlisting}
						// convex_hull_for_binary.cpp

// Copyright 2020 Kustova Anastasiya
#include "../../../modules/task_3/kustova_a_convex_hull_for_binary/convex_hull_for_binary.h"
#include <mpi.h>
#include <vector>
#include <random>
#include <iostream>
#include <ctime>
void printImage(int ** image, int height, int width) {
    for (int i = 0; i < height; i++) {
        for (int j = 0; j < width; j++) {
            std::cout << image[i][j] << " ";
        }
        std::cout << std::endl;
    }
}

int giveNumbersToComponents(int ** image, int height, int width) {
    int size, rank;
    MPI_Comm_rank(MPI_COMM_WORLD, &rank);
    MPI_Comm_size(MPI_COMM_WORLD, &size);
    int componentsNumber = 0;
    const int delta_height = height / size;
    const int rem_height = height % size;
    int start_height;
    if (rank < rem_height) {
        start_height = rank * (delta_height + 1);
    } else {
        start_height = delta_height * rank + rem_height;
    }
    if (image[start_height][0] != 0) {
        componentsNumber++;
        image[start_height][0] = componentsNumber;
    }
    for (int j = 1; j < width; j++) {
        if (image[start_height][j] != 0) {
            if (image[start_height][j - 1] != 0) {
                image[start_height][j] = image[start_height][j - 1];
            } else {
                componentsNumber++;
                image[start_height][j] = componentsNumber;
            }
        }
    }
    int rem_count;
    if (rank < rem_height) {
        rem_count = delta_height + 1;
    } else {
        rem_count = delta_height;
    }
    rem_count += start_height;
    for (int i = start_height + 1; i < rem_count; i++) {
        for (int j = 0; j < width; j++) {
            if (image[i][j] != 0) {
                if ((j > 0) && (image[i - 1][j - 1] != 0)) {
                    image[i][j] = image[i - 1][j - 1];
                } else if (image[i - 1][j] != 0) {
                    image[i][j] = image[i - 1][j];
                } else if ((j < width - 1) && (image[i - 1][j + 1] != 0)) {
                    image[i][j] = image[i - 1][j + 1];
                } else if ((j > 0) && (image[i][j - 1] != 0)) {
                    image[i][j] = image[i][j - 1];
                } else {
                    componentsNumber++;
                    image[i][j] = componentsNumber;
                }
            }
        }
    }
    if (size > 1) {
        std::vector<int>nums(size);
        MPI_Allgather(&componentsNumber, 1, MPI_INT, &nums[0], 1, MPI_INT, MPI_COMM_WORLD);
        int step = 0;
        componentsNumber = 0;
        for (int i = 0; i < size; i++) {
            if (i < rank)
            step += nums[i];
            componentsNumber += nums[i];
        }
        if (rank != 0) {
            for (int i = start_height; i < rem_count; i++) {
                for (int j = 0; j < width; j++) {
                    if (image[i][j] != 0) {
                        image[i][j] += step;
                    }
                }
            }
        }
        int k = 0;
        std::vector<int> send(height*width);
        for (int i = 0; i < height; i++) {
            for (int j = 0; j < width; j++) {
                send[k] = image[i][j];
                k++;
            }
        }
        std::vector<int> recv(height*width);
        MPI_Allreduce(&send[0], &recv[0], height*width, MPI_INT, MPI_MAX, MPI_COMM_WORLD);
        k = 0;
        for (int i = 0; i < height; i++) {
            for (int j = 0; j < width; j++) {
                image[i][j] = recv[k];
                k++;
            }
        }
    }
    return componentsNumber;
}
std::vector<int> combineComponentParts(int ** image, int height, int width, int componentsNumber) {
    int size, rank;
    MPI_Comm_rank(MPI_COMM_WORLD, &rank);
    MPI_Comm_size(MPI_COMM_WORLD, &size);
    int* graf = new int[componentsNumber * componentsNumber];
    if (rank == 0) {  
        for (int i = 0; i < componentsNumber * componentsNumber; i++) {
                graf[i] = 0;
            }        
        for (int i = 1; i < height; i++) {
            for (int j = 0; j < width; j++) {
                if (image[i][j] != 0) {
                    if (j > 0) {
                        if (image[i - 1][j - 1] != 0 && image[i - 1][j - 1] != image[i][j]) {
                            int g1 = image[i - 1][j - 1] - 1;
                            int g2 = image[i][j] - 1;
                            graf[g1 * componentsNumber + g2] = 1;
                            graf[g2 * componentsNumber + g1] = 1;
                        }
                        if (image[i][j - 1] != 0 && image[i][j - 1] != image[i][j]) {
                            int g1 = image[i][j - 1] - 1;
                            int g2 = image[i][j] - 1;
                            graf[g1 * componentsNumber + g2] = 1;
                            graf[g2 * componentsNumber + g1] = 1;
                        }
                    }
                    if (j < width - 1) {
                        if (image[i - 1][j + 1] != 0 && image[i - 1][j + 1] != image[i][j]) {
                            int g1 = image[i - 1][j + 1] - 1;
                            int g2 = image[i][j] - 1;
                            graf[g1 * componentsNumber + g2] = 1;
                            graf[g2 * componentsNumber + g1] = 1;
                        }
                        if (image[i][j + 1] != 0 && image[i][j + 1] != image[i][j]) {
                            int g1 = image[i][j + 1] - 1;
                            int g2 = image[i][j] - 1;
                            graf[g1 * componentsNumber + g2] = 1;
                            graf[g2 * componentsNumber + g1] = 1;
                        }
                    }
                    if (image[i - 1][j] != 0 && image[i - 1][j] != image[i][j]) {
                        int g1 = image[i - 1][j] - 1;
                        int g2 = image[i][j] - 1;
                        graf[g1 * componentsNumber + g2] = 1;
                        graf[g2 * componentsNumber + g1] = 1;
                    }
                }
            }
        }
        int k;
        std::vector<int>vec;
        for (int i = 0; i < componentsNumber; i++) {
            k = 0;
            vec.clear();
            for (int j = 0; j < componentsNumber; j++) {
                if (graf[i * componentsNumber + j] != 0) {
                    k++;
                    vec.push_back(j);
                }
            }
            if (k > 1) {
                for (unsigned int t = 0; t < vec.size(); t++) {
                    for (unsigned int p = 0; p < vec.size(); p++) {
                        graf[vec[t] * componentsNumber + vec[p]] = 1;
                    }
                }
            }
        }
    }
    MPI_Bcast(&graf[0], componentsNumber * componentsNumber, MPI_INT, 0, MPI_COMM_WORLD);
    std::vector<int> send(height*width);
    int k = 0;
    for (int h = 0; h < height; h++) {
        for (int w = 0; w < width; w++) {
            send[k] = image[h][w];
            k++;
        }
    }
    int delta_height = height / size;
    int rem_height = height % size;
    int start_height;
    int rem_count;
    if (rank < rem_height) {
        start_height = rank * (delta_height + 1);
        rem_count = delta_height + 1;
    } else {
        start_height = delta_height * rank + rem_height;
        rem_count = delta_height;
    }
    rem_count += start_height;
    for (int i = componentsNumber - 2; i >= 0; i--) {
        for (int j = componentsNumber - 1; j >= i + 1; j--) {
            if (graf[i * componentsNumber + j] != 0) {
                for (int h = start_height; h < rem_count; h++) {
                    for (int w = 0; w < width; w++) {
                        if (image[h][w] == j + 1) {
                            image[h][w] = i + 1;
                            send[h * width + w] = image[h][w];
                        }
                    }
                }

            }
        }
    }
    std::vector<int> recv(height*width);
    MPI_Allreduce(&send[0], &recv[0], height*width, MPI_INT, MPI_MIN, MPI_COMM_WORLD);
    k = 0;
    for (int h = 0; h < height; h++) {
        for (int w = 0; w < width; w++) {
            image[h][w] = recv[k];
            k++;
        }
    } 
    std::vector<int> components;
    for (int h = 0; h < height; h++) {
        for (int w = 0; w < width; w++) {
            if (image[h][w] != 0) {
                if (components.empty()) {
                    components.push_back(image[h][w]);
                } else {
                    bool ok = true;
                    int csize = components.size();
                    for (int i = 0; i < csize; i++) {
                        if (components[i] == image[h][w]) {
                            ok = false;
                            break;
                        }
                    }
                    if (ok) {
                        components.push_back(image[h][w]);
                    }
                }
            }
        }
    }
    return components;
}

std::vector<int> searchConnectedComponents(int ** image, int height, int width) {
    int componentsNumber = giveNumbersToComponents(image, height, width);
    std::vector<int> components = combineComponentParts(image, height, width, componentsNumber);
    return components;
}


int ** imageOr(int ** image1, int ** image2, int height, int width) {
    int size, rank;
    MPI_Comm_rank(MPI_COMM_WORLD, &rank);
    MPI_Comm_size(MPI_COMM_WORLD, &size);
    int delta_height = height / size;
    int rem_height = height % size;
    int start_height;
    int max_height;
    if (rank < rem_height) {
        start_height = rank * (delta_height + 1);
        max_height = delta_height + 1;
    } else {
        start_height = delta_height * rank + rem_height;
        max_height = delta_height;
    }
    max_height += start_height;

    int * resultImage = new int[height * width];
    for (int i = 0; i < height * width; i++) {
        resultImage[i] = 0;
    }

    for (int i = start_height; i < max_height; i++) {
        for (int j = 0; j < width; j++) {
            if (image1[i][j] != 0) {
                resultImage[i * width + j] = image1[i][j];
            } else if (image2[i][j] != 0) {
                resultImage[i * width + j] = image2[i][j];
            } else {
                resultImage[i * width + j] = 0;
            }
        }
    }
    std::vector<int> recv(height*width);
    MPI_Allreduce(&resultImage[0], &recv[0], height*width, MPI_INT, MPI_MAX, MPI_COMM_WORLD);
    int k = 0;
    for (int i = 0; i < height; i++) {
        for (int j = 0; j < width; j++) {
            image1[i][j] = recv[k];
            k++;
        }
    }
    return image1;
}



int ** buildComponentConvexHull(int ** image, int height, int width, int type) {
    std::vector<Point> vec;
    for (int i = 0; i < height; i++) {
        for (int j = 0; j < width; j++) {
            if (image[i][j] != 0) {
                Point a;
                a.x = i;
                a.y = j;
            vec.push_back(a);
            }
        }
    }
    std::vector<Point> res;
    if (type == 0) {
        res = buildHull(vec);
    } else {
        res = buildHull_sequential(vec);
    }
    for (unsigned int i = 0; i < res.size(); i++) {
        int x1 = (int)(res[i].x);
        int y1 = (int)(res[i].y);
        image[x1][y1] = 3;  
    }
    if (vec.size() != 1) {
        image = drawLines(image, height, width,res);
    } 
    return image;    
}

int ** getComponent(int ** image, int height, int width, int number) {
    int ** resultImage = new int*[height];
    for (int i = 0; i < height; i++) {
        resultImage[i] = new int[width];
        for (int j = 0; j < width; j++) {
            if (image[i][j] == number) {
                resultImage[i][j] = 1;
            } else {
                resultImage[i][j] = 0;
            }
        }
    }
    return resultImage;
}

int ** buildImageConvexHull(int ** image, int height, int width, int type) {
    int ** resultImage = new int*[height];
    int rank;
    MPI_Comm_rank(MPI_COMM_WORLD, &rank);
    int ** comp = new int*[height];
    for (int i = 0; i < height; i++) {
        resultImage[i] = new int[width];
        comp[i] = new int[width];
        for (int j = 0; j < width; j++) {
            resultImage[i][j] = 0;
            comp[i][j] = image[i][j];
        }
    }
    std::vector<int> components = searchConnectedComponents(comp, height, width);
    while (!components.empty()) {
        int component = components.back();
        components.pop_back();
        int ** result = buildComponentConvexHull(getComponent(comp, height, width, component), height, width, type);
        resultImage = imageOr(resultImage, result, height, width);
    }
    return resultImage;
}

bool is_counter_clockwise(const Point& p, const Point& i, const Point& q) {
  return (i - p).x * (q - p).y - (q - p).x * (i - p).y < 0;
}

Point find_leftmost_point(std::vector<Point> buf, int rank, int size, int countpr) {
    Point most_point;
    if (countpr == 0 || rank == 0) {
        most_point = buf[0];
    }
    for (unsigned int i = 1; i < buf.size(); i++) {
        if (buf[i].x < most_point.x) {
            most_point = buf[i];
        }
    }
    std::vector<Point> points(size);
    if (countpr == 0) {
        MPI_Gather(&most_point, 2, MPI_INT, &points[0], 2, MPI_INT, 0, MPI_COMM_WORLD);
    }

    if (rank == 0) {
        most_point = points[0];
        for (unsigned int i = 1; i < points.size(); i++) {
            if (points[i].x < most_point.x) {
                most_point = points[i];
            }
        }
    }
    MPI_Bcast(&most_point, 2, MPI_INT, 0, MPI_COMM_WORLD);
    return most_point;
}

std::vector<Point> buildHull(std::vector<Point> m_set) {
    int rank, size;
    int countpr = 0;    //0 - work all process, 1 - work one process
    MPI_Comm_rank(MPI_COMM_WORLD, &rank);
    MPI_Comm_size(MPI_COMM_WORLD, &size);
    Point leftmost_point;
    int delta = m_set.size() / size;
    int rem = m_set.size() % size;
    if (delta == 0) countpr = 1;
    std::vector<Point> buffer(delta);
    MPI_Scatter(&m_set[rem], delta * 2, MPI_INT, &buffer[0], delta * 2, MPI_INT, 0, MPI_COMM_WORLD);
    if (rank == 0) {
        buffer.insert(buffer.begin(), m_set.begin(), m_set.begin() + rem);
    }
    leftmost_point = find_leftmost_point(buffer, rank, size, countpr);
    std::vector<Point> hull;
    std::vector<Point> points(size);
    Point endpoint;
    do {
        hull.push_back(leftmost_point);
        endpoint = m_set.at(0);
        if (countpr == 0 || rank == 0) {
            for (auto i = buffer.begin(); i < buffer.end(); i++) {
                if (is_counter_clockwise(hull.back(), *i, endpoint) || endpoint == leftmost_point) {
                    endpoint = *i;
                }
            }
        }
        if (countpr == 0) {
            MPI_Gather(&endpoint, 2, MPI_INT, &points[0], 2, MPI_INT, 0, MPI_COMM_WORLD);
        }
        if (rank == 0) {
            for (auto point : points) {
                if (is_counter_clockwise(hull.back(), point, endpoint) || endpoint == leftmost_point) {
                    endpoint = point;
                }
            }
        }
        MPI_Bcast(&endpoint, 2, MPI_INT, 0, MPI_COMM_WORLD);
        leftmost_point = endpoint;
    } while (endpoint != hull.at(0));
    return hull;
 }

std::vector<Point> buildHull_sequential(std::vector<Point> m_set) {
    Point leftmost_point = m_set[0];
    for (unsigned int i = 1; i < m_set.size(); i++) {
        if (m_set[i].x < leftmost_point.x) {
            leftmost_point = m_set[i];
        }
    }
    std::vector<Point> hull;
    Point endpoint;
    do {
        hull.push_back(leftmost_point);
        endpoint = m_set.at(0);
        for (auto i : m_set) {
           if (is_counter_clockwise(hull.back(), i, endpoint) || endpoint == leftmost_point) {
               endpoint = i;
            }
        }
        leftmost_point = endpoint;
    } while (endpoint != hull.at(0));
    return hull;
}

int ** drawLines(int ** image, int height, int width, std::vector<Point> vec)
{
    int ** result = image;
    int x1, y1, x2, y2, start, end;
    double func, min;
    int x_min, y_min;
    for (unsigned int i = 0; i < vec.size(); i++) {
        x1 = vec[i].x;
        y1 = vec[i].y;
        if (i == vec.size() - 1) {
            x2 = vec[0].x;
            y2 = vec[0].y;
        } else {
        x2 = vec[i+1].x;
        y2 = vec[i + 1].y;
        }
        if (my_abs(x1 - x2) <= 1) {
            if (y1 < y2) {
                start = y1;
                end = y2;
            } else {
                start = y2;
                end = y1;    
            }
            for (int k = start; k < end; k++) {
                result[x1][k] = 2;
            }
            continue;            
        } else if (x1 < x2) {
            start = x1;
            end = x2;
        } else {
            start = x2;
            end = x1;
        }
        for (int j = start + 1; j < end; j++) {
            min = width*width;
            for (int k = 0; k < width; k++) {
                func = sqrt((x1 - j) * (x1 - j) + (y1 - k)*(y1 - k)) + sqrt((x2 - j) * (x2 - j) + (y2 - k)*(y2 - k));
                if (func <= min) {
                    min = func;
                    x_min = j;
                    y_min = k;
                }
                
            }
            if (result[x_min][y_min] != 3)
             result[x_min][y_min]= 2;
        }
    }
    return result;
}
double my_abs(double x) {
    if (x > 0) {
        return x;
    } else {
        return -x;
    }
}
\end{lstlisting}
    \end{document}


