\documentclass{report}

\usepackage[T2A]{fontenc}
\usepackage[utf8]{luainputenc}
\usepackage[english, russian]{babel}
\usepackage[pdftex]{hyperref}
\usepackage[14pt]{extsizes}
\usepackage{listings}
\usepackage{color}
\usepackage{geometry}
\usepackage{enumitem}
\usepackage{multirow}
\usepackage{graphicx}
\usepackage{indentfirst}

\geometry{a4paper,top=2cm,bottom=3cm,left=2cm,right=1.5cm}
\setlength{\parskip}{0.5cm}
\setlist{nolistsep, itemsep=0.3cm,parsep=0pt}

\lstset{language=C++,
    basicstyle=\footnotesize,
    keywordstyle=\color{blue}\ttfamily,
    stringstyle=\color{red}\ttfamily,
    commentstyle=\color{green}\ttfamily,
    morecomment=[l][\color{magenta}]{\#},
    tabsize=4,
    breaklines=true,
    breakatwhitespace=true,
    title=\lstname,
}

\makeatletter
\renewcommand\@biblabel[1]{#1.\hfil}
\makeatother

\begin{document}

\begin{titlepage}

\begin{center}
Министерство науки и высшего образования Российской Федерации
\end{center}

\begin{center}
Федеральное государственное автономное образовательное учреждение высшего образования \\
Национальный исследовательский Нижегородский государственный университет им. Н.И. Лобачевского
\end{center}

\begin{center}
Институт информационных технологий, математики и механики
\end{center}

\vspace{4em}

\begin{center}
\textbf{\LargeОтчет по лабораторной работе} \\
\end{center}
\begin{center}
\textbf{\Large«Линейная фильтрация изображений (блочное разбиение). Ядро Гаусса 3x3»} \\
\end{center}

\vspace{4em}

\newbox{\lbox}
\savebox{\lbox}{\hbox{text}}
\newlength{\maxl}
\setlength{\maxl}{\wd\lbox}
\hfill\parbox{7cm}{
\hspace*{5cm}\hspace*{-5cm}\textbf{Выполнил:} \\ студент группы 381808-2 \\ Крень Полина\\
\\
\hspace*{5cm}\hspace*{-5cm}\textbf{Проверил:}\\ доцент кафедры МОСТ, \\ кандидат технических наук \\ Сысоев А. В.\\
}
\vspace{\fill}

\begin{center} Нижний Новгород \\ 2020 \end{center}

\end{titlepage}

\setcounter{page}{2}

\newpage

% Введение
\section*{Введение}
\addcontentsline{toc}{section}{Введение}
Под фильтрацией изображений понимают операцию, имеющую своим результатом изображение того же размера, полученное из исходного по некоторым правилам (фильтрам). Обычно цвет каждого пикселя результирующего изображения обусловлена цветами пикселей, расположенных в некоторой его окрестности в исходном изображении.
\par
Методы линейной фильтрации предназначены для улучшения качества зашумленных изображений с сохранением перепадов яркости. Значительная часть алгоритмов фильтрации описывается локальным оператором свёртки.
\par
Целью данной работы является реализация параллельной линейной фильтрации для изображения, заданного в оттенках серого, и сравнение ее производительности с последовательной реализацией.
\newpage

% Постановка задачи
\section*{Постановка задачи}
\addcontentsline{toc}{section}{Постановка задачи}
В рамках лабораторной работы нужно реализовать параллельный алгоритм линейной фильтрации изображений (ядро гаусса 3x3).
\par
Требуется выполнить следующее: последовательный алгоритм линейной фильтрации, параллельный алгоритм линейной фильтрации (блочное разбиение), провести вычислительные эксперименты, сравнить время работы параллельной и последовательной реализаций.
\newpage

% Метод решения
\section*{Метод решения}
\addcontentsline{toc}{section}{Метод решения}
Пусть задано исходное изображение A, и обозначим интенсивности его пикселей A(x, y). Линейный фильтр определяется функцией F, заданной на растре. Данная функция называется ядром фильтра, а сама фильтрация производится при помощи операции дискретной свертки (взвешенного суммирования): 
\par
B(x, y) =  F(i, j) * A(x + i, y + j)
\par
Суммирование производится по (i, j), и значение каждого пикселя B(x, y) определяется пикселями изображения A, которые лежат в окне N, центрированном в точке (x, y) (будем обозначать это множество N(x, y)). Ядро фильтра, заданное на прямоугольной окрестности N, может рассматриваться как матрица m на n, где длины сторон являются нечетными числами. В данной задаче рассматривается ядро 3x3.
\par 
Если пиксель (x, y) находится в окрестности краев изображения. В этом случае                 A(x + i, y + j) в может соответствовать пикселю A, лежащему за пределами изображения A. Данную проблему можно разрешить несколькими способами:
\par 
1) Не проводить фильтрацию для таких пикселей, обрезав изображение B по краям или закрасив их, к примеру, черным цветом. 
\par 
2) Не включать соответствующий пиксель в суммирование, распределив его вес F(i, j) равномерно среди других пикселей окрестности N(x, y). 
\par 
3) Доопределить значения пикселей за границами изображения при помощи экстраполяции.
\par
4) Доопределить значения пикселей за границами изображения, при помощи зеркального отражения.
\par
В данной реализации линейной фильтрации был применен второй пункт.
\newpage

% Схема распараллеливания
\section*{Схема распараллеливания}
\addcontentsline{toc}{section}{Схема распараллеливания}
В начале все процессы коммуникатора вычисляют свою часть матрицы (начальные строка и столбец, количество строк и столбцов). 
\par
Деление матрицы происходит следующим образом: делим количество столбцов и количество строк на корень из количества процессов, остатки от деления распределяем между процессами.
\par
После деления матрицы каждый процесс (кроме нулевого) отправляет нулевому свои начальные строку и столбец, количество строк и столбцов. Это осуществляется с помощью Send и Resv. Нулевой процесс собирает все данные в один вектор для дальнейшего удобства вставки в результирующую матрицу.
\par
Далее все процессы получают на вход матрицу целиком. Это осуществляется с помощью Bcast. Каждый процесс вычисляет свои значения для новой матрицы и отправляет нулевому процессу (Send и Resv). В конце нулевой процесс все полученные результаты складывает в нужные позиции результирующей матрицы.
\par
В итоге результирующая матрица будет расположена в нулевом процессе.
\newpage

% Описание программной реализации
\section*{Описание программной реализации}
\addcontentsline{toc}{section}{Описание программной реализации}
В программе реализованы методы:
\begin{itemize}
\item Генерирование рандомной матрицы с соответствующими размерами (диапазон значений: от 0 до 255): 
\end{itemize}
\begin{lstlisting}
	std::vector <int> getImg(int rows, int cols);
\end{lstlisting}
\begin{itemize}
\item Вычисление значения для конкретного пикселя результирующей матрицы:
\end{itemize}
\begin{lstlisting}
	int compute_pixel(const std::vector <int>& mask, int mask_size, int norm, const std::vector <int>& mtx, int i, int j, int rows, int cols);
\end{lstlisting}
\begin{itemize}
\item Выполнение параллельной линейной фильтрации: 
\end{itemize}
\begin{lstlisting}
	std::vector <int> parall_linear_filter(const std::vector <int> mask, std::vector <int> a, int rows, int cols);
\end{lstlisting}
\begin{itemize}
\item Выполнение последовательной линейной фильтрации: 
\end{itemize}
\begin{lstlisting}
	std::vector <int> SequentialLinearFilter(const std::vector<int> mask, std::vector <int> a, int rows, int cols);
\end{lstlisting}

\newpage
% Результаты экспериментов
\section*{Результаты экспериментов}
\addcontentsline{toc}{section}{Результаты экспериментов}
Эксперименты проводились на 4-х ядерном компьютере с операционной системой – Windows 10. Результаты экспериментов приведены в таблице:
\begin{table}[!h]
\caption{Результаты вычислительных экспериментов}
\centering
\begin{tabular}{lllll}
Количество процессов & Паралл. время & Послед. время & Разница\\
2  & 2.322 & 3.203 & 0.881\\
3 & 1.954 & 2.949 & 0.995\\
4  & 2.763 & 3.96 & 1.197\\
\end{tabular}
\end{table}
\par
В ходе эксперимента было доказано, что алгоритм реализован эффективно, ускорение практически линейное. При увеличении количества процессов, ускорение с каждым разом уменьшается. Это объясняется увеличением накладных расходов.
\par
Kорректность работы программы могут результаты гугл-тестов.
\newpage

% Заключение
\section*{Заключение}
\addcontentsline{toc}{section}{Заключение}
В ходе работы были реализованы последовательный и параллельный алгоритмы линейной фильтрации изображений (ядро Гаусса 3x3). Из результатов вычислительных экспериментов видим, что параллельный алгоритм эффективнее последовательного, так как последовательный алгоритм обрабатывает всю матрицу целиком.
\newpage

% Список литературы
\begin{thebibliography}{1}
\addcontentsline{toc}{section}{Список литературы}
\bibitem{Gergel}
Гергель В. П., Стронгин Р. Г. Основы параллельных вычислений для многопроцессорных вычислительных систем. – 2003.
\bibitem{parallel} Parallel: •	Цифровая обработка сигналов. Обработка изображений. Фильтрация изображений. [Электронный ресурс] // URL: https://studfile.net/preview/5830083/page:4/

\end{thebibliography}
\newpage

% Приложение
\section*{Приложение}
\addcontentsline{toc}{section}{Приложение}
\begin{lstlisting}
						// gaussian_kernel.h
// Copyright 2020 Kren Polina
#ifndef MODULES_TASK_3_KREN_P_GAUSSIAN_KERNEL_GAUSSIAN_KERNEL_H_
#define MODULES_TASK_3_KREN_P_GAUSSIAN_KERNEL_GAUSSIAN_KERNEL_H_

#include <vector>

std::vector <int> get_Img(int cols, int rows);
int compute_pixel(const std::vector<int> mtx, const std::vector<int> mask, int i, int j, int rows, int cols);
std::vector <int> parall_linear_filter(const std::vector <int> mask, const std::vector <int> a, int rows, int cols);
std::vector <int> SequentialLinearFilter(const std::vector <int> mask, const std::vector <int> a, int rows, int cols);

#endif  // MODULES_TASK_3_KREN_P_GAUSSIAN_KERNEL_GAUSSIAN_KERNEL_H_

\end{lstlisting}
\begin{lstlisting}
						// gaussian_kernel.cpp

// Copyright 2020 Kren Polina
#include "../../../modules/task_3/kren_p_gaussian_kernel/gaussian_kernel.h"
#include <mpi.h>
#include <algorithm>
#include <ctime>
#include <numeric>
#include <random>
#include <string>
#include <vector>

std::vector<int> get_Img(int rows, int columns) {
  if (rows <= 0 || columns <= 0) {
    throw - 1;
  }
  std::vector<int> matrix(rows * columns);
  std::mt19937 generator(std::time(0));
  std::uniform_int_distribution<> distribution{0, 255};
  for (int i = 0; i < rows * columns; i++) {
    matrix[i] = distribution(generator);
  }
  return matrix;
}

int compute_pixel(const std::vector<int> matrix, const std::vector<int> mask,
                  int i, int j, int rows, int columns) {
  int pixel = 0;
  int mask_size = sqrt(mask.size());
  for (int d = -mask_size / 2; d <= mask_size / 2; ++d) {
    for (int k = -mask_size / 2; k <= mask_size / 2; ++k) {
      int y = j + k;
      int x = i + d;
      if (y < 0 || y > columns - 1) {
        y = j;
      }
      if (x < 0 || x > rows - 1) {
        x = i;
      }
      if (x * columns + y >= columns * rows) {
        x = i;
        y = j;
      }
      pixel += matrix[x * columns + y] * mask[(d + 1) * 3 + k + 1];
    }
  }
  int normal = std::accumulate(mask.begin(), mask.end(), 0);
  if (normal != 0) {
    pixel = std::min(pixel / normal, 255);
  }
  return pixel;
}

std::vector<int> parall_linear_filter(const std::vector<int> mask,
                                      std::vector<int> matrix, int rows,
                                      int columns) {
  if (static_cast<int>(matrix.size()) != rows * columns) {
    throw - 1;
  }

  int size, rank;
  MPI_Comm_size(MPI_COMM_WORLD, &size);
  MPI_Comm_rank(MPI_COMM_WORLD, &rank);

  int mask_size = mask.size();
  int i_ = 0;
  int j_ = 0;
  int rows_ = 0;
  int columns_ = 0;
  int sqrt_size = floor(sqrt(size));
  float sqrt_size_rest = sqrt(size) - sqrt_size;
  int quantity_cols = columns / sqrt_size;
  int rest_cols = columns % sqrt_size;
  int quantity_rows = 0;
  int rest_rows = 0;

  if (sqrt_size_rest == 0) {  
    quantity_rows = rows / sqrt_size; 
    rest_rows = rows % sqrt_size;  
  } else if (fabs(sqrt_size - sqrt_size_rest) <
             1) {  
    quantity_rows = rows - static_cast<int>((rows / (sqrt_size + 0.5))) *
                               sqrt_size;  
    rest_rows = static_cast<int>(rows / (sqrt_size + 0.5)) -
                quantity_rows;  
  } else {
    quantity_rows = rows / (sqrt_size + 1); 
    rest_rows = rows % (sqrt_size + 1); 
  }
  int rule = 0;
  if ((size - (rows / quantity_rows - 1) * sqrt_size) == 0) {
    rule = 1;
  } else {
    rule =
        columns / (size - (rows / quantity_rows - 1) * sqrt_size) >= mask_size
            ? 1
            : 0;
  }
  if (quantity_rows >= mask_size && quantity_cols >= mask_size && size > 1 &&
      rule == 1) {
    MPI_Bcast(matrix.data(), rows * columns, MPI_INT, 0, MPI_COMM_WORLD);
    int count = size - (rows / quantity_rows - 1) * sqrt_size;
    if (rank >= (rows / quantity_rows - 1) * sqrt_size && sqrt_size_rest != 0 &&
        size > 2) {
      sqrt_size = size - (rows / quantity_rows - 1) * sqrt_size;
      quantity_cols = columns / count;
      rest_cols = columns % count;
      i_ = (rows / quantity_rows - 1) * quantity_rows;
      j_ = (sqrt_size - (size - rank)) * quantity_cols;
      if ((size - rank) % sqrt_size < rest_cols) {
        columns_ = 1;
      }
      if (j_ > 0 && rest_cols > 0) {
        j_ += (sqrt_size - (size - rank) - 1);
        if ((sqrt_size - (size - rank) - 1) == 0) {
          j_ += rest_cols;
        }
      }
      rows_ = quantity_rows + rest_rows;
      columns_ += quantity_cols;
    } else {
      i_ = (rank / (sqrt_size)) * quantity_rows;
      if (rank % sqrt_size < rest_cols) {
        j_ = rank % sqrt_size;
        columns_ = 1;
      } else {
        j_ = rest_cols;
      }
      j_ += (rank % sqrt_size) * quantity_cols;
      if (rank >= size - count) {
        rows_ = rest_rows;
      }
      rows_ += quantity_rows;
      columns_ += quantity_cols;
    }
  } else {
    i_ = 0;
    i_ = 0;
    i_ = 0;
    i_ = 0;
    i_ = 0;
    i_ = 0;
    i_ = 0;
    i_ = 0;
    i_ = 0;
    j_ = 0;
    rows_ = rows;
    columns_ = columns;
  }

  std::vector<int> result(rows * columns);

  if (rank == 0) {
    for (int i = i_; i < i_ + rows_; i++) {
      for (int j = j_; j < j_ + columns_; j++) {
        result[i * columns + j] =
            compute_pixel(matrix, mask, i, j, rows, columns);
      }
    }
    if (quantity_rows >= mask_size && quantity_cols >= mask_size && size > 1 &&
        rule == 1) {
      int proc = 0;
      int amount_cols_ = quantity_cols, rest_cols_ = rest_cols;
      MPI_Status status;
      int sqrt_size_ = sqrt_size;
      int i1 = 0, l_ = 0;
      int flag_ = 0, flag = 1;
      if (sqrt_size == 1) {
        i1++;
        flag = 0;
        flag_ = 1;
        l_++;
      }
      for (int i = i1; i < rows / quantity_rows;
           i++) {  
        for (int j = 0; j < quantity_rows; j++) {
          if (proc * sqrt_size + flag_ >=
                  (rows / quantity_rows - 1) * sqrt_size &&
              sqrt_size_rest != 0 && size > 2) {
            amount_cols_ =
                columns / (size - (rows / quantity_rows - 1) * sqrt_size);
            rest_cols_ =
                columns % (size - (rows / quantity_rows - 1) * sqrt_size);
            sqrt_size_ = size - (rows / quantity_rows - 1) * sqrt_size;
          }
          int l1 = 0;
          if (flag == 1) {
            l1++;
          }
          for (int l = l1; l < sqrt_size_; l++) {
            int rest = 0, c = 0;
            if (l >= rest_cols_) {
              rest = rest_cols_;
            } else {
              rest = l;
              c = 1;
            }
            MPI_Recv(&result[(i * quantity_rows + j) * columns +
                             l * amount_cols_ + rest],
                     amount_cols_ + c, MPI_INT, proc * sqrt_size + l + l_, 1,
                     MPI_COMM_WORLD, &status);
          }
        }
        flag = 0;
        amount_cols_ = quantity_cols;
        rest_cols_ = rest_cols;
        sqrt_size_ = sqrt_size;
        proc++;
        if (proc == size - 1) {
          break;
        }
      }
      amount_cols_ = quantity_cols;
      rest_cols_ = rest_cols;
      for (int i = 0; i < rest_rows; i++) {  
        for (int l = 0; l < size - (rows / quantity_rows - 1) * sqrt_size;
             l++) {
          if (proc * sqrt_size + l >= (rows / quantity_rows - 1) * sqrt_size &&
              sqrt_size_rest != 0 && size > 2) {
            amount_cols_ =
                columns / (size - (rows / quantity_rows - 1) * sqrt_size);
            rest_cols_ =
                columns % (size - (rows / quantity_rows - 1) * sqrt_size);
          }
          int rest = 0, c = 0;
          ;
          if (l >= rest_cols_) {
            rest = rest_cols_;
          } else {
            c = 1;
            rest = l;
          }
          MPI_Recv(&result[(rows - rest_rows + i) * columns + l * amount_cols_ +
                           rest],
                   amount_cols_ + c, MPI_INT,
                   (rows / quantity_rows - 1) * sqrt_size + l, 1,
                   MPI_COMM_WORLD, &status);
        }
      }
    }
  } else {
    std::vector<int> result_;
    if (quantity_rows >= mask_size && quantity_cols >= mask_size && rule == 1) {
      for (int i = i_; i < i_ + rows_; i++) {
        for (int j = j_; j < j_ + columns_; j++) {
          result_.push_back(compute_pixel(matrix, mask, i, j, rows, columns));
        }
        MPI_Send(result_.data(), columns_, MPI_INT, 0, 1, MPI_COMM_WORLD);
        result_.clear();
      }
    }
  }
  return result;
}

std::vector<int> SequentialLinearFilter(const std::vector<int> mask,
                                        std::vector<int> matrix, int rows,
                                        int columns) {
  if (static_cast<int>(matrix.size()) != rows * columns) {
    throw - 1;
  }
  std::vector<int> result;
  for (int i = 0; i < rows; i++) {
    for (int j = 0; j < columns; j++) {
      result.push_back(compute_pixel(matrix, mask, i, j, rows, columns));
    }
  }
  return result;
}
\end{lstlisting}
\begin{lstlisting}
						// main.cpp

// Copyright 2020 Kren Polina
#include <gtest-mpi-listener.hpp>
#include <vector>
#include "../../../modules/task_3/kren_p_gaussian_kernel/gaussian_kernel.h"

TEST(Linear_Filtering_MPI, Test_not_correct_size_of_matrix) {
    int rank;
    MPI_Comm_rank(MPI_COMM_WORLD, &rank);
    if (rank == 0) {
        ASSERT_ANY_THROW(get_Img(-1, 3));
    }
}

TEST(Linear_Filtering_MPI, Test_not_correct_solve_1) {
    std::vector <int> res;
    int rank;
    std::vector <int> mask = { 0, 0, 0, 0, 1, 0, 0, 0, 0 };
    MPI_Comm_rank(MPI_COMM_WORLD, &rank);
    if (rank == 0) {
        res = get_Img(5, 7);
        ASSERT_ANY_THROW(parall_linear_filter(mask, res, 3, 2));
    }
}

TEST(Linear_Filtering_MPI, Test_correct_solve_2) {
    int rows = 3, columns = 3;
    std::vector <int> mask = { 0, 1, 0, 1, -4, 1, 0, 1, 0 };
    std::vector <int> res(rows * columns), res_seq(rows * columns);
    int rank;
    MPI_Comm_rank(MPI_COMM_WORLD, &rank);
    if (rank == 0) {
        res[0] = 150;
        res[1] = 128;
        res[2] = 100;
        res[3] = 200;
        res[4] = 175;
        res[5] = 100;
        res[6] = 50;
        res[7] = 250;
        res[8] = 200;
        res_seq[0] = 28;
        res_seq[1] = 41;
        res_seq[2] = 28;
        res_seq[3] = -225;
        res_seq[4] = -22;
        res_seq[5] = 175;
        res_seq[6] = 350;
        res_seq[7] = -325;
        res_seq[8] = -50;
    }
    std::vector <int> ans(rows * columns);
    ans = parall_linear_filter(mask, res, rows, columns);
    if (rank == 0) {
        ASSERT_EQ(res_seq, ans);
    }
}

TEST(Linear_Filtering_MPI, Test_correct_solve_3) {
    int rows = 10, columns = 5;
    std::vector <int> mask = { 1, 2, 1, 2, 4, 2, 1, 2, 1 };
    std::vector <int> result(rows * columns);
    int rank;
    MPI_Comm_rank(MPI_COMM_WORLD, &rank);
    if (rank == 0) {
        result = get_Img(rows, columns);
    }
    std::vector <int> ans(rows * columns);
    double start = MPI_Wtime();
    ans = parall_linear_filter(mask, result, rows, columns);
    double time = MPI_Wtime() - start;
    std::vector <int> ans_seq(rows * columns);
    if (rank == 0) {
      std::cout << std::setprecision(16) << " Time paral:" << time << std::endl;
      double start2 = MPI_Wtime();
      ans_seq = SequentialLinearFilter(mask, result, rows, columns);
      double time2 = MPI_Wtime() - start2;
      std::cout << std::setprecision(16) << " Time seq:" << time2 << std::endl;
        ASSERT_EQ(ans_seq, ans);
    }
}

TEST(Linear_Filtering_MPI, Test_correct_solve_4) {
    int rows = 200, cols = 400;
    std::vector <int> mask = { 1, 2, 1, 2, 4, 2, 1, 2, 1 };
    std::vector <int> result(rows * cols);
    int rank;
    MPI_Comm_rank(MPI_COMM_WORLD, &rank);
    if (rank == 0) {
        result = get_Img(rows, cols);
    }
    std::vector <int> ans(rows * cols);
    double start = MPI_Wtime();
    ans = parall_linear_filter(mask, result, rows, cols);
    double time = MPI_Wtime() - start;
    std::vector <int> ans_seq(rows * cols);
    if (rank == 0) {
      std::cout << std::setprecision(16) << " Time paral:" << time << std::endl;
      double start2 = MPI_Wtime();
        ans_seq = SequentialLinearFilter(mask, result, rows, cols);
      double time2 = MPI_Wtime() - start2;
      std::cout << std::setprecision(16) << " Time seq:" << time2 << std::endl;
        ASSERT_EQ(ans_seq, ans);
    }
}

int main(int argc, char** argv) {
    ::testing::InitGoogleTest(&argc, argv);
    MPI_Init(&argc, &argv);

    ::testing::AddGlobalTestEnvironment(new GTestMPIListener::MPIEnvironment);
    ::testing::TestEventListeners& listeners =
        ::testing::UnitTest::GetInstance()->listeners();

    listeners.Release(listeners.default_result_printer());
    listeners.Release(listeners.default_xml_generator());

    listeners.Append(new GTestMPIListener::MPIMinimalistPrinter);
    return RUN_ALL_TESTS();
}

\end{lstlisting}
    \end{document}