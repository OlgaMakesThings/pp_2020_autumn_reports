\documentclass{report}

\usepackage[T2A]{fontenc}
\usepackage[utf8]{luainputenc}
\usepackage[english, russian]{babel}
\usepackage[pdftex]{hyperref}
\usepackage[14pt]{extsizes}
\usepackage{listings}
\usepackage{color}
\usepackage{geometry}
\usepackage{enumitem}
\usepackage{multirow}
\usepackage{graphicx}
\usepackage{indentfirst}

\geometry{a4paper,top=2cm,bottom=3cm,left=2cm,right=1.5cm}
\setlength{\parskip}{0.5cm}
\setlist{nolistsep, itemsep=0.3cm,parsep=0pt}

\lstset{language=C++,
	basicstyle=\footnotesize,
	keywordstyle=\color{blue}\ttfamily,
	stringstyle=\color{red}\ttfamily,
	commentstyle=\color{green}\ttfamily,
	morecomment=[l][\color{magenta}]{\#}, 
	tabsize=4,
	breaklines=true,
	breakatwhitespace=true,
	title=\lstname,       
}

\makeatletter
\renewcommand\@biblabel[1]{#1.\hfil}
\makeatother

\begin{document}
	
	\begin{titlepage}
		
		\begin{center}
			Министерство науки и высшего образования Российской Федерации
		\end{center}
		
		\begin{center}
			Федеральное государственное автономное образовательное учреждение высшего образования \\
			Национальный исследовательский Нижегородский государственный университет им. Н.И. Лобачевского
		\end{center}
		
		\begin{center}
			Институт информационных технологий, математики и механики
		\end{center}
		
		\vspace{4em}
		
		\begin{center}
			\textbf{\LargeОтчет по лабораторной работе} \\
		\end{center}
		\begin{center}
			\textbf{\Large«Вычисление многомерных интегралов методом трапеций»} \\
		\end{center}
		
		\vspace{4em}
		
		\newbox{\lbox}
		\savebox{\lbox}{\hbox{text}}
		\newlength{\maxl}
		\setlength{\maxl}{\wd\lbox}
		\hfill\parbox{7cm}{
			\hspace*{5cm}\hspace*{-5cm}\textbf{Выполнил:} \\ студент группы 381806-1 \\ Пронин И. П.\\
			\\
			\hspace*{5cm}\hspace*{-5cm}\textbf{Проверил:}\\ Нестеров А. Ю.\\
		}
		\vspace{\fill}
		
		\begin{center} Нижний Новгород \\ 2020 \end{center}
		
	\end{titlepage}
	
	\setcounter{page}{2}
	
	% Содержание
	\tableofcontents
	\newpage
	
	% Введение
	\section*{Введение}
	\addcontentsline{toc}{section}{Введение}
	Задача вычисления определенного интеграла I для некоторой заданной на отрезке [a, b] функции f(x) является классической задачей математического анализа.
	\par Зачастую приходится использовать приближенные формулы для вычисления определенного интеграла на основе значений подынтегральной функции f(x), так как нахождение значения интеграла с помощью формулы Ньютона - Лейбница $\int_a^b f(x)dx$ = F(b) - F(a) чревато некоторыми вычислительными сложностями. Такие специальные приближенные формулы называют квадратурными формулами или формулами численного интегрирования.
	\par Большинство методов численного интегрирования, объединены общим принципом: область интегрирования разбивается по каждой из осей на равное количество частей. В каждой из полученных маленьких областей интеграл приближается простой функцией (например, линейной), значение которой вычисляется явно (путем вычисления подынтегрального выражения в одной или нескольких точках). Ввиду линейности оператора интегрирования по областям полученные значения суммируются и представляют собой результат интегрирования.
	\par К данному типу относятся одномерные метод прямоугольников, метод трапеций, метод парабол (метод Симпсона). Перечисленные методы обобщается и для многомерного случая.
	
	\newpage
	
	% Постановка задачи
	\section*{Постановка задачи}
	\addcontentsline{toc}{section}{Постановка задачи}
	Основной задачей проекта является изучение метода трапеций для решения множественных интегралов различной сложности, а также разработка программы, решающей подобные определённые интегралы. 
	Подзадачи: 
	\par•	Реализовать последовательный алгоритм метода трапеций для вычисления интеграла 
	\par•	Реализовать параллельный алгоритм метода трапеций для вычисления интеграла (данный алгоритм подразумевает распараллеливание задачи на заданное число процессов с целью увеличения производительности). 
	\par•	Протестировать алгоритмы с помощью тестов, реализованных на  Google C++ Testing Framework 
	\par•	Сравнить время работы последовательного и параллельного алгоритмов
	
	\newpage
	
	% Описание алгоритма
	\section*{Описание алгоритма}
	\addcontentsline{toc}{section}{Описание алгоритма}
	Предположим, что нам нужно приближенно вычислить определенный интеграл $\int_a^b f(x)dx$, подынтегральная функция которого y = f(x) непрерывна на отрезке [a;b]. Для этого разделим отрезок [a;b] на несколько равных интервалов длины h точками  a=x0<x1<x2<...<xn-1<xn=b. Обозначим количество полученных интервалов как n.
	\par Найдем шаг разбиения: h = (b-a)/n. Определим узлы из равенства xi=a+i * h, i=0, 1,..., n.
	\par На элементарных отрезках рассмотрим подынтегральную функцию [xi-1; xi], i=1, 2,.., n.
	\par Возьмем выражение(f(xi-1)+f(xi))/2 * h в качестве приближенного значения интеграла $\int_{xi-1}^{xi} f(x)dx$. Т.е. примем $\int_{xi-1}^{xi} f(x)dx$ = (f(xi-1)+f(xi))/2 * h.
	\newpage
	
	% Описание схемы распараллеливания
	\section*{Описание схемы распараллеливания}
	\addcontentsline{toc}{section}{Описание схемы распараллеливания}
	В лабораторной работе осуществляется распараллеливание многомерного интеграла на заданное число процессов. 
	\par Каждый процесс получает пределы интегрирования функции, величину h, заранее посчитанную на нулевом процессе, а так же количество отрезков разбиения. Затем исходя из количества отрезков разбиения и количества выделенных процессов рассчитывается количество точек, приходящееся на каждый процесс. Каждый процесс в свою очередь находит свои точки и рассчитывает значения функции в этих точках. В самом конце значения функции со всех процессов складываются и умножаются на величину h. 
	\newpage
	
	% Описание программной реализации
	\section*{Описание программной реализации}
	\addcontentsline{toc}{section}{Описание программной реализации}
	Программа состоит из заголовочного файла trapeze.h, в котором объявлены функции:
	\begin{lstlisting}
	double SequentialOperations(double(*function)(std::vector<double>), std::vector<double>, std::vector<double>, int)\end{lstlisting} - вычисление многомерного интеграла методом трапеций, последовательная версия
    \begin{lstlisting}
	double ParllelOperations(double(*function)(std::vector<double>), std::vector<double>, std::vector<double>, int)\end{lstlisting} - вычисление многомерного интеграла методом трапеций, параллельная версия 
	\par файла trapeze.cpp в котором содержится реализация данных функций и файла main.cpp в котором содержаться тесты для проверки корректности программы.
	\newpage
	
	% Подтверждение корректности
	\section*{Подтверждение корректности}
	\addcontentsline{toc}{section}{Подтверждение корректности}
	Корректность работы программы подтверждается пятью тестами:
	\begin{lstlisting}
		TEST(ParallelOperationsMPI, function1)
	\end{lstlisting} - взятие двумерного интеграла от функции (x + y) по области x : [1, 2], y : [5, 6] c числом разбиений равным 100. 
	\begin{lstlisting}
		TEST(ParallelOperationsMPI, function2)
	\end{lstlisting}
	- взятие двумерного интеграла от функции (x * x + y) по области x : [-1, 5], y : [2, 6] c числом разбиений равным 100. 
	\begin{lstlisting}
	TEST(ParallelOperationsMPI, function3)
	\end{lstlisting}
	 - взятие трехмерного интеграла от функции (x + y + z) по области x : [-1, 5], y : [2, 6], z : [3, 10] c числом разбиений равным 100. 
	\begin{lstlisting}
	TEST(ParallelOperationsMPI, function4)
    \end{lstlisting}
	 - взятие трехмерного интеграла от функции (x * x + y + z * z) по области x : [-1, 5], y : [2, 12], z : [3, 20] c числом разбиений равным 100.
	\begin{lstlisting}
	TEST(ParallelOperationsMPI, function5)
    \end{lstlisting}
	 - взятие двумерного интеграла от функции (cos(x) + sin(y)) по области x : [-1, 5], y : [11, 32] c числом разбиений равным 2000.  
	\par Все тесты проходят проверку, что является доказательством корректной работы программы.
	\newpage
	
	% Результаты экспериментов
	\section*{Результаты экспериментов}
	\addcontentsline{toc}{section}{Результаты экспериментов}
	Эксперименты проводились при взятии  двумерного интеграла от функции (cos(x) + sin(y)) по области x : [-1, 5], y : [11, 32] c числом разбиений равным 2000 на четырехъядерном процессоре.   
	\begin{table}[!h]
		{Результаты экспериментов}
		\centering
		\begin{tabular}{lllll}
			Количество процессов &Последовательный алгоритм, с.  &Параллельный алгоритм, с. \\
			2        & 0.470852           & 0.242908       \\
			4        & 0.453265           & 0.110226       \\
			8       & 0.336512           & 0.092426       \\     
			
		\end{tabular}
	\end{table}
	\par Как видно при увеличении числа процессов наблюдается значительное ускорение работы алгоритма. На восьми процессах такого значительного прироста ускорения не происходит из-за ограничений, связанных с характеристиками процессора. 
	\par По результатам можно сделать вывод, что параллельное выполнение программы выигрывает во времени у последовательного.
	
	\newpage
	
	% Заключение
	\section*{Заключение}
	\addcontentsline{toc}{section}{Заключение}
    В ходе работы реализованы последовательная и параллельная версия алгоритма трапеций для вычисления многомерных интегралов. А также выполнены все поставленные подзадачи. 
	\newpage
	
	% Список литературы
	\begin{thebibliography}{1}
		\addcontentsline{toc}{section}{Список литературы}
		\bibitem{1}	Самарский А.А.  
		«Введение в численные методы»  
		Книга написана на основе курса лекций, читавшихся автором па факультете вычислительной математики и кибернетики МГУ  1982 год  272 страницы 
		
		\bibitem{2} Метод трапеций (zaochnik.com)
	\end{thebibliography}
	\newpage
	
	% Приложение
	\section*{Приложение}
	\addcontentsline{toc}{section}{Приложение}
	В данном разделе находится код, написанный в рамках лабораторной работы.
	\begin{lstlisting}
		// trapeze.h
		
		// Copyright 2020 Pronin Igor
		#ifndef MODULES_TASK_3_PRONIN_I_INTEGRAL_TRAPEZE_TRAPEZE_H_
		#define MODULES_TASK_3_PRONIN_I_INTEGRAL_TRAPEZE_TRAPEZE_H_
		#include <mpi.h>
		#include <vector>
		double SequentialOperations(double(*function)(std::vector<double>), std::vector<double>, std::vector<double>, int);
		double ParllelOperations(double(*function)(std::vector<double>), std::vector<double>, std::vector<double>, int);
		#endif  // MODULES_TASK_3_PRONIN_I_INTEGRAL_TRAPEZE_TRAPEZE_H_
	\end{lstlisting}
	\begin{lstlisting}
		//trapeze.cpp
		
	// Copyright 2020 Pronin Igor
	#include <mpi.h>
	#include <vector>
	#include <cmath>
	#include "../../../modules/task_3/pronin_i_integral_trapeze/trapeze.h"
	double SequentialOperations(double(*function)(std::vector<double>),
	std::vector<double> a, std::vector<double> b, int n) {
		size_t mer = a.size();
		double result = 0;
		std::vector<double> h(mer);
		std::vector<double> points(mer);
		for (size_t i = 0; i < mer; i++)
		h[i] = (b[i] - a[i]) / n;
		int count = pow(n, mer);
		for (int i = 0; i < count; i++) {
			for (size_t j = 0; j < mer; j++)
			points[j] = a[j] + h[j] * (i % n + 0.5);
			double f = function(points);
			result = result + f;
		}
		for (size_t i = 0; i < mer; i++)
		result = result * h[i];
		return result;
	}
	
	double ParllelOperations(double(*function)(std::vector<double>),
	std::vector<double> a, std::vector<double> b, int n) {
		int size, rank;
		MPI_Comm_size(MPI_COMM_WORLD, &size);
		MPI_Comm_rank(MPI_COMM_WORLD, &rank);
		size_t mer = a.size();
		std::vector<double> h(mer);
		std::vector<double> points(mer);
		
		std::vector<double> acopy(mer);
		std::vector<double> bcopy(mer);
		int ncopy = n;
		
		int count = 0;
		if (rank == 0) {
			for (size_t i = 0; i < mer; i++) {
				h[i] = (b[i] - a[i]) / n;
				acopy[i] = a[i];
				bcopy[i] = b[i];
			}
			count = pow(n, mer);
		}
		MPI_Bcast(&count, 1, MPI_INT, 0, MPI_COMM_WORLD);
		MPI_Bcast(&ncopy, 1, MPI_INT, 0, MPI_COMM_WORLD);
		MPI_Bcast(&h[0], mer, MPI_DOUBLE, 0, MPI_COMM_WORLD);
		MPI_Bcast(&acopy[0], mer, MPI_DOUBLE, 0, MPI_COMM_WORLD);
		MPI_Bcast(&bcopy[0], mer, MPI_DOUBLE, 0, MPI_COMM_WORLD);
		
		int ostatok = count % size;
		int step = count / size;
		int rankstep = 0;
		double result = 0.0;
		
		if (rank == 0) {
			rankstep = step + ostatok;
			for (int i = 0; i < rankstep; i++) {
				for (size_t j = 0; j < mer; j++)
				points[j] = a[j] + h[j] * (i % n + 0.5);
				double f = function(points);
				result = result + f;
			}
		} else {
			rankstep = rank * step + ostatok;
			for (int i = 0; i < step; i++) {
				for (size_t j = 0; j < mer; j++)
				points[j] = a[j] + h[j] * (rankstep % n + 0.5);
				double f = function(points);
				result = result + f;
				rankstep++;
			}
		}
		double itog = 0.0;
		MPI_Reduce(&result, &itog, 1, MPI_DOUBLE, MPI_SUM, 0, MPI_COMM_WORLD);
		for (size_t i = 0; i < mer; i++)
		itog = itog * h[i];
		return itog;
	}
	\end{lstlisting}
	\begin{lstlisting}
		//main.cpp
		
		// Copyright 2020 Pronin Igor
		#include <gtest-mpi-listener.hpp>
		#include <gtest/gtest.h>
		#include <mpi.h>
		#include <vector>
		#include <cmath>
		#include "./trapeze.h"
		
		double function1(std::vector<double> node) {
			double x = node[0];
			double y = node[1];
			return (x + y);
		}
		double function2(std::vector<double> node) {
			double x = node[0];
			double y = node[1];
			return (x * x + y);
		}
		double function3(std::vector<double> node) {
			double x = node[0];
			double y = node[1];
			double z = node[2];
			return(x + y + z);
		}
		double function4(std::vector<double> node) {
			double x = node[0];
			double y = node[1];
			double z = node[2];
			return(x * x + y + z * z);
		}
		double function5(std::vector<double> node) {
			double x = node[0];
			double y = node[1];
			return(cos(x) + sin(y));
		}
		
		TEST(Parallel_Operations_MPI, function1) {
			int rank;
			MPI_Comm_rank(MPI_COMM_WORLD, &rank);
			int mer = 2;
			int n = 100;
			std::vector<double> a(mer);
			std::vector<double> b(mer);
			a = { 1, 2 };
			b = { 5, 6 };
			double presult = ParllelOperations(function1, a, b, n);
			if (rank == 0) {
				double sresult = SequentialOperations(function1, a, b, n);
				double error = 0.0001;
				ASSERT_NEAR(presult, sresult, error);
			}
		}
		
		TEST(Parallel_Operations_MPI, function2) {
			int rank;
			MPI_Comm_rank(MPI_COMM_WORLD, &rank);
			int mer = 2;
			int n = 100;
			std::vector<double> a(mer);
			std::vector<double> b(mer);
			a = { -1, 2 };
			b = { 5, 6 };
			double presult = ParllelOperations(function2, a, b, n);
			if (rank == 0) {
				double sresult = SequentialOperations(function2, a, b, n);
				double error = 0.0001;
				ASSERT_NEAR(presult, sresult, error);
			}
		}
		
		TEST(Parallel_Operations_MPI, function3) {
			int rank;
			MPI_Comm_rank(MPI_COMM_WORLD, &rank);
			int mer = 2;
			int n = 100;
			std::vector<double> a(mer);
			std::vector<double> b(mer);
			a = { 5, 2, 3 };
			b = { -1, 6, 10 };
			double presult = ParllelOperations(function3, a, b, n);
			if (rank == 0) {
				double sresult = SequentialOperations(function3, a, b, n);
				double error = 0.0001;
				ASSERT_NEAR(presult, sresult, error);
			}
		}
		
		TEST(Parallel_Operations_MPI, function4) {
			int rank;
			MPI_Comm_rank(MPI_COMM_WORLD, &rank);
			int mer = 2;
			int n = 100;
			std::vector<double> a(mer);
			std::vector<double> b(mer);
			a = { 5, 2, 3 };
			b = { -1, 12, 20 };
			double presult = ParllelOperations(function4, a, b, n);
			if (rank == 0) {
				double sresult = SequentialOperations(function4, a, b, n);
				double error = 0.0001;
				ASSERT_NEAR(presult, sresult, error);
			}
		}
		
		TEST(Parallel_Operations_MPI, function5) {
			int rank;
			MPI_Comm_rank(MPI_COMM_WORLD, &rank);
			int mer = 2;
			int n = 100;
			std::vector<double> a(mer);
			std::vector<double> b(mer);
			a = { -1, 11 };
			b = { 5, 32 };
			double presult = ParllelOperations(function5, a, b, n);
			if (rank == 0) {
				double sresult = SequentialOperations(function5, a, b, n);
				double error = 0.0001;
				ASSERT_NEAR(presult, sresult, error);
			}
		}
		
		int main(int argc, char** argv) {
			::testing::InitGoogleTest(&argc, argv);
			MPI_Init(&argc, &argv);
			
			::testing::AddGlobalTestEnvironment(new GTestMPIListener::MPIEnvironment);
			::testing::TestEventListeners& listeners =
			::testing::UnitTest::GetInstance()->listeners();
			
			listeners.Release(listeners.default_result_printer());
			listeners.Release(listeners.default_xml_generator());
			
			listeners.Append(new GTestMPIListener::MPIMinimalistPrinter);
			return RUN_ALL_TESTS();
		}
		
		int main(int argc, char* argv[]) {
			::testing::InitGoogleTest(&argc, argv);
			MPI_Init(&argc, &argv);
			
			::testing::AddGlobalTestEnvironment(new GTestMPIListener::MPIEnvironment);
			::testing::TestEventListeners& listeners =
			::testing::UnitTest::GetInstance()->listeners();
			
			listeners.Release(listeners.default_result_printer());
			listeners.Release(listeners.default_xml_generator());
			
			listeners.Append(new GTestMPIListener::MPIMinimalistPrinter);
			return RUN_ALL_TESTS();
		}
	\end{lstlisting}
\end{document}